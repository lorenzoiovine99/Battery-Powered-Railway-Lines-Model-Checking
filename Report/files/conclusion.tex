In this report we presented an analysis of the problem of the Model Checking of Battery Powered Railway Lines.
The aim was to check a realistic configuration in order to understand the validity of the model.\\
From the results showed, we succeeded in finding a configuration where all the properties to be checked are satisfied.
We also tried to increase verification performances by reducing the state space, in order to decrease processing time.
The main characteristic that emerged are the recharge policies. The real bottleneck of the system is the fact that
there are stations with a number of tracks lower than trains in the system. It is important to find a consistent recharge
policy in order to efficiently describe the behavior of the system without letting the parameters to be inconsistent.\\
This project allowed us to improve our team working and knowledge about the subjects of the course.
