\subsection{Purpose}
The fight to reduce greenhouse gas emissions is bringing together researchers and manufacturers from all over the world.
In particular, rechargeable batteries as a source of power in place of fossil fuels are already widespread in cars and
making their way into the rail transport sector. Battery-powered trains are already operative in several countries like
Japan, Austria, and Britain. Italy is also planning on producing and deploying fully-electric trains starting mid-2022,
thanks to a deal with Hitachi Rail.\\
\\ Like any electric vehicle, trains can cover a limited distance running only on battery power before needing to recharge.
In this project, we will model a railway line in which electric trains can either recharge in a station. Nevertheless,
trains must still reach the following station on time; in case of excessive delay, the company is obliged to issue
monetary compensation to the passengers.\\
\\ Precisely, given a set of simplifying assumptions, we will model the main actors of the system as a network of \textbf{Timed Automata (TA)}
whose behavior depends on specific key parameters.


\bigskip
\subsection{High Level Model Description}
We created two different configurations for the railway model. Both of them include 4 trains and 3 stations.\\
The first one represents the main configuration of the system and verifies all the properties. The railway model is set as follows:
\\
\begin{figure}[H]
    \centering
    \includegraphics[scale=0.4]{images/poweredRailway.png}
\end{figure}
\bigskip

The second configuration doesn't verify the mandatory properties of the project and it is set as follows:
\\
\begin{figure}[H]
    \centering
    \includegraphics[scale=0.4]{images/poweredRailwaySimple.png}
\end{figure}
\bigskip

\subsection{Initializations}
The stations have the following initial configurations:
\begin{itemize}
    \item \textbf{Station 0: } 2 tracks, 1 available
    \item \textbf{Station 1: } 3 tracks, 2 available
    \item \textbf{Station 2: } 2 tracks, 0 available
\end{itemize}
\newpage

The trains that we designed have constant speed set to 120km/h. They are initialized as follows:
\begin{itemize}
    \item \textbf{Train 0 - charge 100: } starts from station 0 with station 2 as destination
    \item \textbf{Train 1 - charge 100: } starts from station 1 with station 2 as destination
    \item \textbf{Train 2 - charge 100: } starts from station 2 with station 0 as destination
    \item \textbf{Train 3 - charge 100: } starts from station 2 with station 0 as destination
\end{itemize}
\bigskip

\subsection{Design assumptions}
In order to efficiently describe the model, we decided to make the following assumptions:
\begin{itemize}
    \item Every station has less tracks than the total number of train.
    \item A clock unit is equal to a minute.
    \item For each train the destination is the last station, except for the one that start from the last one who has
    as destination the first station.
    \item The lower bound to allow passengers to get on and off the train is of 4 clock unit.
    \item We incluce a charging multiplier and two different discharging multiplier, one for the waiting and one for the travel.
\end{itemize}

\bigskip

\subsection{Design Choices}
\begin{itemize}
    \item We decided not to design the railway with a dedicated template. That's because the railway's most important features
    are implicitly designed and verified, without creating additional variables.
    \item In order to save time when checking properties we avoid redundant clocks for operations that are not issued in parallel.
    \item Our \emph{Recharge Policy} is based on a control made in function \textit{chargingTime} in the \textit{Train Template},
    that allows to recharge the train at least for the lower bound (described before). In case the train needs more time to 
    recharge in order to get to the next station, the time spent in the station is the mean value between the lower bound
    and the upper bound (calculated as follows: $MaximumDelay-\frac{DistanceToNextStation}{trainSpeed}$) in order not 
    to overcome the maximum delay. We thought that this is a good compromise between charging the train and have some delay 
    in reaching the next station.
    \item In order to model the station, the distances and the maximum delays between stations we used two matrices. It is enough 
    to change the number of trains/stations, initialize them and update the matrices, and the system will "adapt" to the 
    new configuration.
    \item All the variables that could be declared as constant, are declared as constant, in order to save time when checking properties.
\end{itemize}

